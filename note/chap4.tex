\section{Convexity} % (fold)
\label{sec:convexity}

\subsection{Convex Sets} % (fold)
\label{sub:convex_sets}

\begin{definition}[convex combination]
	Consider $\v{x}_i$,
	\[
\sum_{i=1}^n \lambda_i\v{x}
	\]
	is a convex combination of $\v{x}$ if 
	\[
\lambda_i\ge 0 \;\; \text{and} \;\; \sum_{i=1}^n \lambda_i = 1
	\]
\end{definition}

\begin{definition}[convex set]
	A set C is convex if the line segment joining any two points in the set is contained in the set.
\end{definition}

\begin{example}
	Consider C a vector space. If C is convex then
	\[
\theta \v{x}_i + (1-\theta)\v{x}_2 \in C \;\;\forall \theta
	\]
	if $\v{x}_1, \v{x}_2 \in C$ and $\theta\in[0,1]$.
\end{example}

\begin{example}
	Let
	\[
C=\{\v{x}\;|\;\v{a}^\top\v{x}=b\}	
	\]
	Note that C is a hyperplane. It can be rewritten into
	\begin{align*}
		\v{a}(\v{x}-\v{x}_0) &= 0\\
		\v{a}^\top\v{x} &= \v{a}^\top\v{x}_0 =b
	\end{align*}
	To check wheter C is convex, consider $\v{x}_1,\v{x}_2\in C$ and let
	\[
\v{x}_3 = \theta\v{x}_1+(1-\theta)\v{x}_2
	\]
	We know that
	\[
\v{a}^\top\v{x}_3 = \theta\v{a}^\top\v{x}_1 + (1-\theta)\v{a}^\top\v{x}_2 = b
	\]
	Therefore $\v{x}_3$ belongs to C and C is convex.
\end{example}

\begin{remark}
	A hyperplane (a plane which's dimension is 1 less than the dimension of its ambient space) divides the space into two half spaces. The set
	\[
\{\v{x}\;|\;\v{a}^\top\v{x}\ge b\}
	\]
	defines a hyperplane, where $\v{a}$ is perpendicular to all vectors on this plane. This hyperplane naturally generates a counter part
	\[
\{\v{x}\;|\;\v{a}^\top\v{x}\le b\}
	\]
	Example:
	\[
P = \{\v{x}\;|\;\v{a}^\top(\v{x}-\v{x}_0)\ge 0\} \;\; N = \{\v{x}\;|\;\v{a}^\top(\v{x}-\v{x}_0)\le 0\}
	\]
	devides the space into two parts (P for positive and N for negative).
\end{remark}

\begin{example}
	Consider
	\[
P = \{A\;|\;A\in\mathbb{S}^n,\;\;A\text{ is PSD}\}
	\]
	Recall that A is PSD iff
	\[
\v{x}^\top A\v{x} \ge 0 \;\;\forall\v{x}\in\mathbb{R}^n
	\]
	Is P convex? Let
	\[
A_1,\;A_2 \in P\;\;and\;\;A_3 = \theta A_1+(1-\theta)A_2
	\]
	Then
	\begin{align*}
		\v{x}^\top A_3\v{x} = \theta(\v{x}^\top A_1\v{x}) + (1-\theta)\v{x}^\top A_2\v{x} \ge 0\\
		\implies A_3\in P
	\end{align*}
	Therefore P is convex.
\end{example}

\begin{remark}
	Linear transformations always preserve convexity.
\end{remark}

\begin{theorem}[separating hyperplane theorem]
	Let C, D be convex sets and $C \cap D = \emptyset$. Then there exists hyperplane $\v{a}^\top\v{x} = b$ separating two sets such that
	\begin{align*}
		\forall \v{x}\in C\;\; \v{a}^\top\v{x}\ge b\\
		\forall \v{x}\in D\;\; \v{a}^\top\v{x}\le b\\
	\end{align*}
\end{theorem}

Proof: TODO

% subsection convex_sets (end)

\subsection{Convex Functions} % (fold)
\label{sub:convex_functions}

\begin{definition}[convex functions]
	Let
	\[
f:\mathbb{R}^n\rightarrow\mathbb{R}
	\]
	Function f is convex if the domain of f is a convex set and
	\[
f(\theta\v{x}+(1-\theta)\v{y})\le \theta f(\v{x})+(1-\theta)f(\v{y})\;\; 0\le\theta\le1
	\]
	The above inequality is called \textbf{Jensen's Inequality}. Here is an example of a convex function that visualizes the Jensen's Inequality.

\begin{center}
\begin{tikzpicture}
\begin{axis}[
scale = 0.6,
xmin = 0, xmax = 4,
ymin = 0, ymax = 10,
axis lines* = left,
xtick = {0}, ytick = \empty,
clip = false,
]
% Graph
\addplot[domain = 0:10, restrict y to domain = 0:10, samples = 400, color = red]{(x-2)^2+2};

\addplot[color = black, dashed, thick] coordinates {(1, 3) (1, 0)};
\addplot[color = black, dashed, thick] coordinates {(3.5, 4.25) (3.5, 0)};
\addplot[color = black, dashed, thick] coordinates {(3.5, 4.25) (1, 3)};
\addplot[color = black, dashed, thick] coordinates {(2, 0) (2, 3.5)};

% Points
\filldraw (3.5, 4.25) circle[radius=1.5pt];
\filldraw (1, 3) circle[radius=1.5pt];
\filldraw (2, 2) circle[radius=1.5pt];
\filldraw (2, 3.5) circle[radius=1.5pt];
\node[right=2pt of {(3.5, 4.25)}, outer sep=2pt] {$f(\v{y})$};
\node[left=2pt of {(1, 3)}, outer sep=2pt] {$f(\v{x})$};
\node[above=2pt of {(2, 4)}, outer sep=2pt] {$\theta f(\v{x})+(1-\theta)f(\v{y})$};
\node[below=2pt of {(2, 0))}, outer sep=2pt] {$f(\theta\v{x}+(1-\theta)\v{y})$};
\node[above right=2pt of {(4.7, 8))}, outer sep=2pt] {$f$};

% Labels
\node [right] at (current axis.right of origin) {$x$};
\node [above] at (current axis.above origin) {$y$};
\end{axis}
\end{tikzpicture}
\end{center}

If the "cord" is always above the function, the function is \textbf{convex}. If the "cord" is always below the function, the function is \textbf{concave}.

\end{definition}

\begin{theorem}
	If a function f is convex, any local minimum is the global minimum.
\end{theorem}

\begin{definition}[Epigraph]
	The epigraph of a function f is defined as
	\[
Epi f = \{(x, t)\;|\;x\in dom f \;\; f(x) \le t\}
	\]
	f is a convex function $\iff$ Epi f is a convex set.
\end{definition}

\begin{theorem}[First-order condition]
	Define $f:\mathbb{R}^n\rightarrow\mathbb{R}$ a differentiable function. Then f is convex iff
	\[
f(\v{y})\ge f(\v{x})+\nabla f(\v{x})^\top(\v{y}-\v{x}) \;\; \forall \v{x}, \v{y}\in dom f\;\;0\le\theta\le1
	\]
\end{theorem}

\begin{remark}[Implication of the FOC]
	If $\nabla f(\v{x}_*) = 0$ and f is convex, then
	\begin{align*}
		f(\v{y}) &\ge f(\v{x})+0(\v{y}-\v{x})\\
		f(\v{y}) &\ge f(\v{x})
	\end{align*}
	For all y in the domain, which means that $\v{x}_*$ is a global minimum!!
\end{remark}

\begin{theorem}[Second-order condition]
	Let $f:\mathbb{R}^n\rightarrow\mathbb{R}$ who's domain is convex and is twice-differentiable. f is convex iff
	\[
	\nabla^2f(\v{x}) \succeq 0
	\]
	In another word, $\nabla^2f(\v{x})$ is positive semi-definite.
\end{theorem}

\begin{definition}[Strict Convexity]
	Dom f convex. For all x y in domain, f is strictly convex iff
	\[
f(\theta\v{x}+(1-\theta)\v{y}) < \theta f(\v{x})+(1-\theta)f(\v{y})
	\]
	FOC:
	\[
f(\v{y}) > f(\v{x})+\nabla f(\v{x})^\top(\v{y}-\v{x}) \;\; \forall \v{x}, \v{y}\in dom f\;\; 0<\theta<1
	\]
	SOC:
	\[
\nabla^2f(\v{x}) \succ 0
	\]
\end{definition}

\begin{remark}
	If f is a stright line, f is both convex and concave, but not strictly convex.
\end{remark}

\begin{definition}[Strong Convexity]
	Dom f convex. For all x y in domain, f is $\mu$-strongly convex iff
	\[
f(\v{y})\ge f(\v{x})+\nabla f(\v{x})^\top(\v{y}-\v{x}) + \frac{\mu}{2}\|\v{y}-\v{x}\|^2
	\]
\end{definition}

\begin{remark}[implication of strong convexity]
	Recall that by Taylor's theorem, for $f(\v{x}):=\mathbb{R}^n\rightarrow\mathbb{R}$, the derivative of f is
	\[
f(\v{y}) \approx f(\v{x})+\nabla f^\top(\v{y}-\v{x}) + \frac{1}{2}(\v{y}-\v{x})^\top\nabla^2f(\v{y}-\v{x})
	\]
	If we let $\mu I = \nabla^2f$, we have
	\[
\frac{\mu}{2}\|\v{y}-\v{x}\|^2 = \frac{1}{2}(\v{y}-\v{x})^\top\mu I(\v{y}-\v{x})
	\]
	Thus the implication of strong convexity is that the hessian of f is at least $\mu I$.
\end{remark}

\begin{remark}
	Strong convexity $\implies$ strict convexity $\implies$ convexity
\end{remark}

\begin{remark}
	For matrices A and B, 
	\[
	A \succeq B \implies A-B\succeq 0
	\]
\end{remark}

% subsection convex_functions (end)

% section convexity (end)