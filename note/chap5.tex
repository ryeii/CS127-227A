\section{Gradient Descent} % (fold)
\label{sec:gradient_descent}

\subsection{Introduction to Gradient Descent} % (fold)
\label{sub:introduction_to_gradient_descent}

\begin{definition}[Gradient Descent]
	Gradient Descent is an approach to unconstrained optimization problems. The basic idea is to nudge the function in the right direction by a little bit in every step, and after a lot of steps the function will arrive at a local minimum. Formally, for a step size s and a direction $\v{v}$,
	\[
f(\v{x}+s\v{v}) \approx f(\v{x})+s<\nabla f(\v{x}), \v{v}>
	\]
	Recall Cauchy-Schwartz, the magnitude of $<\nabla f(\v{x}), \v{v}>$ is maximized if $\v{v}$ is aligned with $\nabla f(\v{x})$. We want to minimize the inner product while maximize its magnitude so the function steps towards the minimum at the fastest rate, hence we choose
	\[
\v{v} = -\nabla f(\v{x})
	\]
	The formal algorithm for gradiant descent is defined as follows. Let $\v{x}$ be the parameter of function f. At step k,
	\[
\v{x}_{k+1} = \v{x}_k -\eta\nabla f(\v{x}_k)
	\]
	Where $\v{x}_0$ is the initial point and $\eta$ is the stepsize.
\end{definition}

\begin{example}[GD on LS]
	Let $f(\v{x}) = \|A\v{x}-\v{b}\|_2^2$. It has a direct solution of $\v{x}^* = (A^\top A)^{-1}A^\top\v{b}$. If A is a n*n matrix, the runtime of computing the direct solution is at least $O(n^3)$ (taking a matrix inverse is approx. $O(n^3)$). It is computationally cheaper to use gradient descent. Thus,
	\[
\nabla f(\v{x}) = 2A^\top(A\v{x}-\v{b})
	\]
\begin{align*}
	\v{x}_{k+1} &= \v{x}_k -\eta\nabla f(\v{x}_k)\\
	&= \v{x}_k - \eta 2A^\top(A\v{x}-\v{b})\\
	\v{x}_{k+1} &= (I-2\eta A^\top A)\v{x}_k + 2\eta A^\top\v{b}
\end{align*}
	Next we need to prove that this algorithm will converge. The following is one of the ways to prove convergence. The difference between optimal value and the k-step value is
	\begin{align*}
		\v{x}_{k+1} - (A^\top A)^{-1}A^\top\v{b} &= (I-2\eta A^\top A)\v{x}_k + 2\eta A^\top\v{b} - (A^\top A)^{-1}A^\top\v{b}\\
		&= (I-2\eta A^\top A)\v{x}_k + 2\eta(A^\top A)(A^\top A)^{-1}A^\top\v{b}-(A^\top A)^{-1}A^\top\v{b}\\
		&= (I-2\eta A^\top A)\v{x}_k + (2\eta A^\top A-I)(A^\top A)^{-1}A^\top\v{b}\\
		&= (I-2\eta A^\top A)(\v{x}_k - (A^\top A)^{-1}A^\top\v{b})
	\end{align*}
	Hence if the absolute values of the eigenvalues of $I-2\eta A^\top A$ are strictly less than 1, GD converges for LS.
\end{example}

\begin{example}[GD on LS continued]
	Since we showed in the previous part that
	\[
\v{x}_{k+1} - (A^\top A)^{-1}A^\top\v{b} = (I-2\eta A^\top A)^{k+1}(\v{x}_k - (A^\top A)^{-1}A^\top\v{b})
	\]
	Where $\eta$ (step size) is a parameter of choice and we want to make sure the absolute values of the eigenvalues of $(I-2\eta A^\top A)$ is strictly less than 1, we should choose an appropriate $\eta$ such that the algorithm converges. 
\end{example}

% subsection introduction_to_gradient_descent (end)

\subsection{Gradient Descent for $\mu$-strongly convex L-smooth functions} % (fold)
\label{sub:gradient_descent_for_}

\begin{definition}[Bounds of convex functions]
	Recall the definition of $\mu$-strongly convex: Dom f convex. For all x y in domain, f is $\mu$-strongly convex iff
	\[
f(\v{y})\ge f(\v{x})+\nabla f(\v{x})^\top(\v{y}-\v{x}) + \frac{\mu}{2}\|\v{y}-\v{x}\|^2
	\]
	Intuitively, this means that there exists a quadratic function under f such that this quadratic function is the lower-bound of f, hence the gradient of f is changing fast enough. On the other hand, \textbf{L-smooth} means that there exists a quadratic function such that f is upper-bounded by this function, hence the gradient of f is not changing too fast. Formally,
	\[
f(\v{y})\le f(\v{x})+\nabla f(\v{x})^\top(\v{y}-\v{x}) + \frac{L}{2}\|\v{y}-\v{x}\|_2^2
	\]
\end{definition}

\begin{theorem}
	Let $f:\mathbb{R}^n\rightarrow\mathbb{R}$ and define optimization problem
	\[
\min_{\v{x}\in\mathbb{R}^n}f(\v{x})
	\]
	Let $\v{x}_*$ be the optimal solution to the above problem. Then for GD approach
	\[
\v{x}_{t+1} = \v{x}_t-\eta\nabla f(\v{x}_t)
	\]
	I can choose an $\eta$ such that
	\[
\|\v{x}_{t+1}-\v{x}_*\|_2^2 \le (C)^{t+1}\|\v{x}_0-\v{x}_*\|_2^2
	\]
\end{theorem}

\begin{lemma}
	Let f L-smooth, then
	\[
\|\nabla f(\v{x})\|_2^2\le 2L(f(\v{x})-f(\v{x}_*))
	\]
\end{lemma}

\begin{proof}
	Since $f(\v{x}_*)$ minimum, we have
	\[
f(\v{x}_*) \le f(\v{x})
	\]
	And
	\[
f(\v{x}_*) \le f(\v{x}-\frac{\nabla f(\v{x})}{L})
	\]
Recall the definition of L-smooth:
\[
f(\v{y})\le f(\v{x})+\nabla f(\v{x})^\top(\v{y}-\v{x}) + \frac{L}{2}\|\v{y}-\v{x}\|_2^2
\]
We choose $f(\v{y}) = f(\v{x}-\frac{\nabla f(\v{x})}{L})$, then
\begin{align*}
	f(\v{x}-\frac{\nabla f(\v{x})}{L}) &\le f(\v{x})+\nabla f(\v{x})^\top(-\frac{\nabla f(\v{x})}{L})+\frac{L}{2}\|-\frac{\nabla f(\v{x})}{L}\|_2^2\\
	&\le f(\v{x})-\frac{1}{L}\|\nabla f(\v{x})\|^2+\frac{1}{2L}\|\nabla f(\v{x})\|_2^2\\
	&\le f(\v{x})+\frac{1}{2L}\|\nabla f(\v{x})\|_2^2\\
\end{align*}
Since $f(\v{x}_*) \le f(\v{x}-\frac{\nabla f(\v{x})}{L})$,
\begin{align*}
	f(\v{x}_*) \le f(\v{x})+\frac{1}{2L}\|\nabla f(\v{x})\|_2^2
\end{align*}
\end{proof}

\begin{lemma}
	If f $\mu$-strongly convex,
	\[
	\nabla f(\v{x})^\top(\v{x}_*-\v{x}) \ge f(\v{x}) - f(\v{x}_*) + \frac{\mu}{2}\|\v{x}_*-\v{x}\|_2^2 
	\]
\end{lemma}

\begin{proof}
	Recall the definition of $\mu$-strong convexity: Dom f convex. For all x y in domain, f is $\mu$-strongly convex iff
	\[
f(\v{y})\ge f(\v{x})+\nabla f(\v{x})^\top(\v{y}-\v{x}) + \frac{\mu}{2}\|\v{y}-\v{x}\|^2
	\]
	Let $\v{y} = \v{x}_*$. We have
	\begin{align*}
		f(\v{x}_*) &\ge f(\v{x})+\nabla f(\v{x})^\top(\v{x}_*-\v{x}) + \frac{\mu}{2}\|\v{x}_*-\v{x}\|^2\\
		f(\v{x}_*) - f(\v{x}) - \frac{\mu}{2}\|\v{x}_*-\v{x}\|_2^2 &\ge \nabla f(\v{x})^\top(\v{x}_*-\v{x})\\
		- f(\v{x}_*) + f(\v{x}) + \frac{\mu}{2}\|\v{x}_*-\v{x}\|_2^2 &\le \nabla f(\v{x})^\top(\v{x}_*-\v{x})
	\end{align*}
\end{proof}

Proof of Main Theorem (Theorem 5.5): TODO

% subsection gradient_descent_for_ (end)

\subsection{Introduction to Stochastic Gradient Descent} % (fold)
\label{sub:introduction_to_stochastic_gradient_descent}

\begin{definition}
	Let $f:\mathbb{R}^n\rightarrow\mathbb{R}$ and define optimization problem
	\[
\min_{\v{x}\in\mathbb{R}^n}f(\v{x})
	\]
Assume the function f is of the following form:
\[
f(\v{x}) = \sum_{i=1}^m \frac{1}{m}f_i(\v{x})
\]
E.g. for least squares problem
\[
\frac{1}{2m}\|A\v{x}-\v{b}\|_2^2 = \frac{1}{2m}\sum_{i=1}^m(\v{a}_i^\top\v{x}-\v{b}_i)^2
\]
For
\[
A = \begin{bmatrix}
	\cdots & \v{a}_1^\top & \cdots \\
	&\vdots&\\
	\cdots & \v{a}_n^\top & \cdots \\
\end{bmatrix}
\]
To calculate the solution of such problem using gradient descent would cost too much computation time (order n time complexity to calculate n losses). Thus instead of computing all of the losses, we choose one component of the gradient in each step.
\[
\v{x}_{k+1} = \v{x}_k-\eta_k\nabla f_i(\v{x}_k)
\]
This works because
\begin{align*}
	\mathbb{E}[\nabla f_i(\v{x}_k)] &= \frac{1}{m}\sum_{i=1}^mf_i(\v{x}) \\
	&= \nabla f(\v{x}_k)
\end{align*}
We usually let $\eta$ be time-dependent.
\end{definition}

\begin{example}
	Consider optimization problem
	\[
f(\v{x}) = \frac{1}{m}\sum_{i=1}^m\|\v{x}-\v{p}_i\|_2^2
	\]
	We use SGD to solve this problem. Let step size $\eta = \frac{1}{t}$ and $\v{x}_0 = 0$, then
	\[
\nabla f(\v{x}) = \frac{1}{2}2(\v{x}-\v{p}_i) = \v{x}-\v{p}_i
	\]
	The $\v{x}$ in each step goes like
	\begin{align*}
		\v{x}_0 &= 0\\
		\v{x}_1 &= \v{x}_0 - \frac{1}{1}(\v{x}_0-\v{p}_1) = \v{p}_1\\
		\v{x}_2 &= \v{x}_1 - \frac{1}{2}(\v{x}_1-\v{p}_2) = \frac{\v{p}_1+\v{p}_2}{2}\\
		\v{x}_3 &= \v{x}_2 - \frac{1}{3}(\v{x}_2-\v{p}_3) = \frac{\v{p}_1+\v{p}_2+\v{p}_3}{3}
	\end{align*}
	The future terms of GD can be projected to
	\[
\v{x}_i = \frac{\sum_{j=1}^ip_j}{i}
	\]
\end{example}

% subsection introduction_to_stochastic_gradient_descent (end)

\subsection{Projected Gradient Descent} % (fold)
\label{sub:projected_gradient_descent}

\begin{definition}[projected gradient descent]
	Differ from ordinary gradient descent, projected gradient descent solves for the general GD problem where $\v{x}$ is constrained. Formally, we want to solve the problem
	\[
\min_{\v{x}}f(\v{x})\;\;,\;\;\v{x}\in C
	\]
	Such that
	\[
\v{x}_{k+1} = \prod_C\v{x}_k-\eta\nabla f(\v{x}_k)
	\]
	And
	\[
\prod_C(\v{y}) = \argmin_{\v{\delta}\in C}\|\v{y}_{k+1}-\v{\delta}\|_2^2
	\]
	However, this could still be computationally expensive, therefore is replaced by Conditional Gradient Descent in the next subsection.
\end{definition}

% subsection projected_gradient_descent (end)

\subsection{Conditional Gradient Descent} % (fold)	
\label{sub:conditional_gradient_descent}

\begin{definition}[conditional gradient descent/Frank–Wolfe algorithm]
	Let $\gamma_k$ be a predetermined sequence. Define
	\[
\v{y}_k = \argmin_{\v{y}\in C}\nabla f(\v{x}_k)^\top\v{y}
	\]
	And
	\begin{align*}
		\v{x}_{k+1} &= (1- \gamma_k)\v{x}_k+ \gamma_k\v{y}_k\\
		&= \v{x}_k + \gamma_k(\v{y}_k-\v{x}_k)
	\end{align*}
	Both Frank-Wolfe and PGD are generally computationally expensive, but Frank-Wolfe is sometimes cheaper. Frank-Wolfe also has a nice sparse property but it is out of scope for this class.
\end{definition}

% subsection conditional_gradient_descent (end)

% section gradient_descent (end)